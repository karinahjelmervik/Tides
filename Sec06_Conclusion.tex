\section{Conclusion}

A key factor to modelling of currents in any ocean model is accurate tidal forcing. Regarding tides in regional and fjord models it is common to use tidal forcing from a global or basin wide model like the TPXO or similar global tidal models. Our experience is that these solutions does not produce adequate accurate results when applied at the boundary of a fjord model. This is due to the fact that today's global tidal models does not properly resolve the bathymetry in the shallow waters close to the coast and hence does not provide the correct phase nor amplitude of the tides in cosatal regions. 

We propose a simple method whereby the global or regional tidal forcing may be adjusted. First we run the fjord model with tidal forcing based on the global or regional tidal model results. Then, we run a harmonic analysis of the results at a location close to the boundary where observed time series of water levels are available. Performing the same analysis to the observed water level we are able to compare the simulated and observed water level for each tidal component at that location. Next we compute the ratio between observed and modelled amplitude and a phase difference for each tidal component. This ratio and phase shift is then used to adjust the tidal forcing. The same ratio and phase difference are applied on the amplitudes of both water level and current. The final step is to rerun the model with adjusted tidal forcing. To evaluate the improvement the results of the latter run is compared to observations.

The method is apllied to the Oslofjord, Norway. The model we use is a curvilinear version of the Regional Ocean Model System (ROMS) adapted to Oslofjord. The fjord opening is relatively narrow and the tidal waves propagates almost perpendicular to the fjord opening. Therefore only one station is deemed sufficiant for the observed water level. On wider openings, more stations might be needed. The evaluation of the results of the second run show improvements both in the modelled tidal elevations and phases not only close to the boundary, but also at the very bottom of the Oslofjord. Hence the results suggest that this simple approach is a viable method by which tidal input from coarser global tidal models may be adjusted. 

Finally it is emphasised that the method does not provide perfect match between observed and simulated tides. In this respect it is important to keep in mind that the propagation of the tidal wave from the boundary into the fjord depnds critically on a correct representation of the topography and irregular coastline of the fjord. It is therefore crucial that the fjord model resolves these features with sufficient accuracy. In our case we exploited the curvilinear option in ROMS to get a resolution that was high enough to resolve the many smaller and larger islands of the Oslofjord, and its many narrow sounds and straits. Nevertheless, as with most models, the topography and irregular coastline have to be smoothed to make the model run properly. Thus the representation of these feature are never a true representation of the topography and irregular coastline, and hence may influence the propagation of the tidal wave as it progresses into the fjord. In addition, and in particular in shallower wparts of the fjord the tidal wave propagation is highly influenced by the bottom friction. Thus care should be exercised when constructing the topography of the fjord model in order to make the tidal wave propagation as close to the true propagation as possible, for instance by use of assimilation techniques.
