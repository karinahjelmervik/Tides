\section{Conclusion}

A key factor to modelling currents in any ocean model is accurate tidal forcing. For high resolution regional models, this can be taken from a global or regional atlas of tides, like the TPXO or similar. Our experience is that these solutions might not produce adequate accurate results when applied at the mouth of a fjord in a fjord model. This is because the phase speed of the tidal wave can vary close to the coast, and that the coastline and the depths are not properly resolved in the coarse tidal atlases.

We have proposed a simple method to adjust tidal forcing. First we run the ocean model with tidal forcing based on the global or regional tidal atlas, e.g. the regional Atlantic TPXO with a resolution of $1/30$ degree. Secondly, we run harmonic analysis in order to compare the simulated and observed water level for each tidal component. The ratio between observed and modelled amplitude and a phase difference are computed for each tidal component, and then used to adjust the tidal forcing. The same ratio and phase difference are applied on the amplitudes of both water level and current. Finally, the model is rerun with adjusted tidal forcing and the results are evaluated against observations.

The method is tested with the Regional Ocean Model System (ROMS) on two different fjords in Norway, the Oslofjord and the Saltfjord, which includes the famous Saltstraum maelstrom.
 
The results show improvements in the modelled tidal elevations and phases in both models, and suggest that this simple approach is one way of correcting coarse tidal atlases to yield improved results in a fjord model.
