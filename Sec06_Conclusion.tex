\section{Conclusion}

A key factor to model water level and currents in any ocean model is accurate tidal forcing. Regarding tides in regional and fjord models it is common to use tidal forcing from a global or basin wide model like the TPXO or similar global tidal models. Our experience is that these solutions does not produce adequate accurate results when applied at the boundary of a fjord model. This is due to the fact that today's global tidal models does not properly resolve the bathymetry in shallow waters close to the coast and hence does not provide the correct phase nor amplitude of the tides in coastal regions. 

We propose a simple method whereby the global or basin wide tidal forcing may be adjusted. First we run the fjord model with tidal forcing based on the global or basin wide tidal model results. Then, we run harmonic analyses on the results at a location close to the boundary where observed time series of water levels are available. Performing the same analyses on the observed water level we are able to compare the modelled and observed water level for each tidal component at that location. Next we compute the ratio between observed and modelled amplitude and a phase difference for each tidal component. This ratio and phase shift is then used to adjust the tidal forcing. The same ratios and phase differences are applied on the amplitudes of both water levels and depth integrated currents in the original tidal forcing. The final step is to rerun the model with adjusted tidal forcing. To evaluate the improvements, the results of the latter run are compared to observations.

The method is applied to the Oslofjord, Norway. Here we apply a curvilinear version of the Regional Ocean Model System (ROMS) adapted to the Oslofjord. The fjord opening is relatively narrow and the tidal waves propagates almost perpendicular to the fjord opening. Therefore only one station is deemed sufficient for the observed water level. On wider openings, more stations might be needed. The evaluation of the results of the second run show improvements both in the modelled tidal elevations and phases not only close to the boundary, but also in the inner parts of the fjord. Hence the results suggest that this simple approach is a viable method by which tidal input from coarser global tidal models may be adjusted. 

Finally it is emphasised that the adjusted tidal forcing improves, but does not provide perfect match between observed and simulated tides in the inner area. In this respect it is important to keep in mind that the propagation of the tidal wave from the boundary into the fjord depends critically on a correct representation of the topography and irregular coastline of the fjord. Thus care should be exercised when constructing the topography of the fjord model in order to make the tidal wave propagation as close to the true propagation as possible, for instance by use of assimilation techniques.
