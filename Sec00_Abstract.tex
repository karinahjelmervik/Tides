\begin{abstract}

Tides are one of the dominant driving forces in coastal waters. Due to complex topography with narrow and shallow straights, the tides in the innermost parts of the Norwegian fjords are both shifted in phase and altered in amplitude compared to the tides in the open water outside the fjords.

In order to model the hydrography and currents in the fjords, accurate tidal forcing is crucial. The global tidal forcing is too coarse for fjord modeling since tides vary over short distances close to the coast line. To overcome this problem, the tidal forcing has to be adjusted to fit the actual tides in the area. We have developed a simple method to do this adjustment. First, a high-resolution ocean model is run with the global tidal forcing on the open boundary. Time series of water level is then analysed and compared with observed water level. Based on the comparison a factor for the amplitude and a phase shift is computed and applied to produce adjusted tidal forcing. The same factor and phase shift is used on tidal current forcing as for tidal water level forcing. The model is then rerun with the adjusted tidal forcing.

The method is tested using the Regional Ocean Model System (ROMS) on two different model areas in Norway; the Oslofjord and Saltstraumen. The results show improvements in the modelled tides in both the inner and the outer parts of the fjords. 

\end{abstract}
