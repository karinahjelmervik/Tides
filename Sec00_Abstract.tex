\begin{abstract}

To model currents in a fjord, accurate tidal forcing is of extreme importance. Due to complex topography with narrow and shallow straits, the tides in the innermost parts of a fjord are both shifted in phase and altered in amplitude compared to the tides in the open water outside the fjord.
Commonly, coastal tide information extracted from a global or regional model is used on the boundary of the fjord model.
Since tides vary over short distances in the shallower waters close to the coast, the global and regional tidal forcing is usually too coarse to achieve sufficiently accurate tides in fjords. We present a straightforward method to remedy this problem by simply adjusting the tides to fit the observed tides at the entrance of the fjord. 
To evaluate the method, we present results for the Oslofjord, Norway. The model we use is a curvilinear version of the Regional Ocean Model System (ROMS) adapted for the Oslofjord. We first run the model using raw tidal forcing on its open boundary. Comparing modelled and observed time series of water level at a tidal gauge station close to the open boundary of the Oslofjord model, a factor for the amplitude and a shift in phase is computed. The amplitude factor and the phase shift are then applied to produce adjusted tidal forcing at the open boundary. Next we rerun the fjord model using the adjusted tidal forcing. The results from the two runs are then compared to independent observations inside the fjord in terms of amplitude and phases of the various tidal components, the total tidal water level, and the tidal currents. The results show improvements in the modelled tides in both the outer, and more importantly, in the inner parts of the fjord.

%First, a high-resolution fjord model is run with the global tidal forcing on its open boundary. Time series of water level is then analysed and compared with observed water level. Based on the comparison a factor for the amplitude and a phase shift is computed and applied to produce adjusted tidal forcing in which the same factor and phase shift is used to adjust both water level and currents. Next the fjord model is rerun using the adjusted tidal forcing on the open boundary. The results are then compared to independent water level observations inside the fjord.  

%To evaluate the method we present results using a curvilinear version of the Regional Ocean Model System (ROMS) for two different fjord areas in Norway, namely the Oslofjord and the Saltfjord. The results show improvements in the modelled tides in both the inner and the outer parts of the fjords. 


\end{abstract}
