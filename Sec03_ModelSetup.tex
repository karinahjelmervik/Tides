\section{Model setup}

The Regional Ocean Model System (ROMS) version 3.6 is applied for both fjords as described in \cite{roed16}. ROMS is a free-surface, terrain-following, primitive equations ocean model widely used by the scientific community for a diverse range of applications \cite[]{shchepetkin05,shchepetkin09,haidvogel08}. 

The fjord models are nested into the NorKyst800 model \cite[]{albretsen11} through daily means. \textcolor{Red}{[}The NorKyst800 model covers the whole Norwegian coast with a resolution of 800 meters. Despite NorKyst800s relatively high resolution it is still not fine enough to resolve the highly irregular geometry and topography of most Norwegian fjords (see Fig.~\ref{fig:NorKyst}). - \textcolor{Red}{fjernes?]}

For both fjords high-resolution, curvilinear grids are applied with 42 terrain-following layers in the vertical using a continuous, double stretching function following \cite{shchepetkin09}. Water levels less than two meters are set to zero. Thereby the shallow parts of the fjords are omitted. In order to avoid model instability and/or spurious deep currents the final masked bathymetry is smoothed as described in \cite{roed16}.

%In the horizontal a curvilinear grid with varying grid size is applied (Fig.~\ref{fig:GridSize}). The characteristic grid size, $\Delta s_{i,j}$, is taken as:
%\begin{eqnarray}
%\Delta s_{i,j}^2 \! = \! 
%\left(\!(x_{i+1,j}-x_{i,j})^2\!\! + \!(y_{i+1,j}-y{i,j})^2\! \right)^{\!0.5}\!\!
%\left(\!(x_{i,j+1}-x_{i,j})^2\!\! + \!(y_{i,j+1}-y_{i,j})^2\! \right)^{\!0.5}
%\end{eqnarray}
%Here $i = 1, \cdots, 299$ and $j =1, \cdots, 899$. $x_{i,j}$ and $y_{i,j}$ are the rho-coordinates of element $(i,j)$. (Sp\o rsm\aa l: Er det hensiktsmessig \aa ta med dette?)

%\begin{figure}[!t]
%\centering
%%\includegraphics[width=0.4\textwidth]{Figurer/GridSize}
%\caption{Characteristic grid size in the whole model area.}
%\label{fig:GridSize}
%\end{figure}

\begin{figure}[!t]
\centering
%\includegraphics[width=0.4\textwidth]{Figurer/GridSize}
\caption{NorKyst area \textcolor{Red}{- fjernes?}}
\label{fig:NorKyst}
\end{figure}

The necessary atmospheric input (wind, pressure, temperature, humidity, and cloud cover) is extracted from the AROME-MetCoOp model with a grid resolution of 2.5 km \cite[]{muller2015}. The freshwater discharges to are taken data from a database constructed by use of the hydrological model HBV \cite[]{beldring2003}.

The tidal input in terms of tidal elevations and currents are based on the TPXO Atlantic database \cite[]{egbert02} for the Oslofjord model. Due to technicalities the TPXO 7.2 were applied for the Saltfjord model. The tidal forcing is to coarse for fjord modeling since the tides vary over short distances and a simple interpolation to the model grid might not be sufficient close to the coast line. In order to omit this problem, other fjord models applies tidal amplitude and phase from observed water level close to the boundary \cite[i.e.]{svendsen96,lynge13}. In the ROMS model it is preferable to include the tidal current major amplitude, minor amplitude, phase, and inclination angle. Here we propose a method on how to adjust the global tidal forcing in terms of both elevation and currents, to local ocean models. 
