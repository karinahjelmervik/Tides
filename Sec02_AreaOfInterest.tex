\section{Area of interest}

The Oslofjord is located in Southeast Norway with the capital city of Norway, Oslo, situated in the innermost part of the fjord (Fig.~\ref{fig:area0}). The population surrounding it comprises about 40\% of the Norwegian population, and is by far the most populated area in Norway. It is therefore important to have good knowledge of the conditions in the Oslofjord. 

The fjord is about 100 km long. The width varies from about 50 km at the entrance (~59$^o$N) to about 1-2 km close to the island Oscarsborg (Fig.~\ref{fig:area1}). 
The inner part of the fjord has two branches. 
The eastern branch has a sill, the Dr{\o}bak Sill, located close to the island Oscarsborg. The Dr{\o}ak Sill consists partly of a man made sill, an underwater jetty only 1-2 meters deep extending halfway across the fjord from the western side. Towards the eastern side there is a natural sill of about 20 meters depth. Due its narrowness and shallowness the Dr{\o}bak Sill area is famous for its strong tidal currents, which easily exceeds 1 m/s even though the mean total tidal amplitude is less than 20 cm. 
The western branch, named the Drammensfjord, is almost cut in two by a narrow, shallow, and long sill which is only 11 meters deep, 180 meters wide, and more than 1 km long. This sill causes a relatively strong tidal current, called the Svelvikstraum.
North of these sills the maximum depth is more than 120 meters in both branches. 
The location of these sills, about two thirds into the fjords, makes the Oslofjord peculiar among Norwegian fjords in that most of them have their sill at the entrance.

The presence of many smaller and larger islands gives rise to many narrow sounds, straits, and channels that impede the water exchange, and impact other propagation of tides. Other noteworthy topographic features comprise deeper and more shallow basins (Fig.~\ref{fig:area1}). The highly varying topography is expected to have a significant impact on the propagation of tides.


\begin{figure}[htb]
\centering
\includegraphics[width=0.7\textwidth]{fig_norgeskart}
\caption{The two areas of interest, and their location in Norway. Map from http://www.norgeskart.no.}
\label{fig:area0}
\end{figure}


The water level and motion in the Oslofjord is also impacted by events in the Skagerrak, which lies in the north-eastern part of the North Sea outside of the fjord's southern boundary. Storm surge events with amplitudes of one meter and higher are observed in the fjord and are mostly associated with wind and pressure events in the Skagerrak/North Sea area. The circulation in the Skagerrak is normally counterclockwise with brackish outflow from the Baltic Sea \cite[]{rodhe96,svendsen96}. This flow pattern generates horizontal pressure gradients near the mouth of the Oslofjord \cite[]{baals90}, and by mechanism described by \cite{klinck81} variation in the wind pattern might also generate mean baroclinic flow events that may mask the tidal exchange in the outer Oslofjord.
Finally we note that the Oslofjord has many river outlets giving rise to its baroclinicity. 

The Norwegian Mapping Authority, Hydrographic Service, has 22 tide gauges along the Norwegian coast. Data is available from 1992 up to the current year \cite[]{tide16}. Four of the gauges are placed in the Oslofjord as marked in  Fig.~\ref{fig:area1}.
Current measurements using a bottom-mounted profiling current meter are performed from 18 September to 10 November 2014 at a position near Filtvedt \cite[]{hjelmervik17}.

\begin{figure}[htb]
\centering
\includegraphics[width=0.8\textwidth]{fig_Oslofjorden_area}
\caption{The Oslofjord, Norway. The position of the four tide gauges in Oslo, Oscarsborg, Viker, and Helgeroa are marked with circles. The position of the current measurements close to Filtvedt is marked with a square. The area shown also conveniently shows the computational domain of the Oslofjord model.}
\label{fig:area1}
\end{figure}

