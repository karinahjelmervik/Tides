\section{Area of interest}
Two areas are chosen in this study in order to ensure that the method is robust, the Oslofjord and the Salt- and Skjerstadfjord (Fig.~\ref{fig:area0}).

\begin{figure}[!t]
\centering
\includegraphics[width=0.5\textwidth]{fig_norgeskart}
\caption{The two areas of interest, and their location in Norway. Map from http://www.norgeskart.no.}
\label{fig:area0}
\end{figure}

\begin{figure}[!t]
\centering
\includegraphics[width=0.5\textwidth]{fig_Oslofjorden_area}
\caption{The Oslofjord. The position of the four tide gauges in Oslo, Oscarsborg, Viker, and Helgeroa are marked with a circle. The position of the current measurements are marked with a square.}
\label{fig:area1}
\end{figure}

The Oslofjord is located in the southeastern part of Norway with the capital city of Norway, Oslo, situated in the innermost part of the fjord (Fig.~\ref{fig:area1}). The fjord lies in the most populated area of Norway and it is therefore important to have good knowledge of the conditions in the Oslofjord. It has an interesting flow pattern due to several river outlets, sills, and complex topography. The circulation is also largely affected by the wind and pressure patterns in the North Sea and Skagerrak. 
Even though the mean total tidal elevation is less than 20 cm, the tidal currents are up to 1 m/s due to the narrow straits and sill depths. In connection with storm surge events, amplitudes of 70 cm have been observed. The Oslofjord has an open, southern boundary towards Skagerrak which lies in the north-eastern part of the North Sea. The circulation in the Skagerrak is counterclockwise with brackish outflow from the Baltic Sea \cite[]{rodhe96,svendsen96}. This flow pattern generate horizontal pressure gradients near the mouth of the Oslofjord \cite[]{baals90}, and by mechanisme described by \cite{klinck81} variation in the wind pattern might generate mean baroclinic flow events that mask the tidal exchange in the outer Oslofjord.

The inner part of the fjord has two branches. The western branch, named the Drammensfjord, is almost cut in two by a narrow, shallow, and long sill which is only 11 metes deep, 180 meters wide, and more than 1 km long. This sill causes a relatively strong tidal current, called Svelvikstraumen. The eastern branch also has a sill, the Dr{\o}bak sound. In the Dr{\o}bak sound, close to the island Oscarsborg, there is a partly man made sill which consist of an underwater barrier only 1-2 meters deep and extends halfway across the fjord from the western side. Towards the eastern side there is a natural sill of about 20 meters depth. North of the sills the maximum depth is more than 120 meters in both branches. This makes the Oslofjord peculiar among Norwegian fjords in that most of them have the sill at the entrance to the fjord.

\begin{figure}[!t]
\centering
\includegraphics[width=0.8\textwidth]{fig_Saltstraumen_area}
\caption{The Saltstraumen is a strong current in a narrow passage between the Saltfjord and the Skjerstradfjord. The position of the tide gauge in Bod{\o} is marked.}
\label{fig:area2}
\end{figure}

The Saltfjord is located in the northern part of Norway (Fig.~\ref{fig:area2}). The Saltstraum is the current in the narrow straight separating the Saltfjord from the Skjerstadfjord, and is considered to be one of the world's strongest tidal currents \cite[]{gjevik09}. The strong current that transports nutrient particles support a dense cover of animals such as anemones, sponges, soft corals, hydroids, bivalves, and echinoderms. The rich and densely fouled hard bottom areas in the Saltstraum stand out as areas of special value as a nature type \cite[]{fagerli15}. This fjord system is generally deep, but the depth from the sill to the mouth of the Saltfjord is less than the depth of the inner part, the glacially carved basin of the Skjerstadfjord. The sill in the Saltstraum is about 50 km from the head of the fjord system. The sill depth is 26 m and the depth in the basin inside it is more than 500 m. The inner basin and the sill area are connected by a channel which is much deeper and a little wider than Saltstraumen. The difference between the water level outside (to the west of) and inside (to the east of) Saltstraumen can be up to one meter, causing current speeds of up to 11 m/s \cite[]{eliassen01}.

The Norwegian Mapping Authority, Hydrographic Service has 22 tide gauges along the Norwegian coast. Data is available from 1992 up to the current year. Four of the gauges are placed in the Oslofjord  (Fig.~\ref{fig:area1}) and one close to the open boundary of the Saltfjord (Fig.~\ref{fig:area2}). Current measurements are performed at a position near Filtvedt in the Oslofjord using a bottom-mounted profiling current meter from 18 September to 25 November 2014.

