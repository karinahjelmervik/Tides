
\section{Area of interest}
Two areas are chosen in this study in order to investigate if the method handles different types of fjords, the Oslo fjord and the Salt- and Skjerstad fjord.

\begin{figure}[!t]
\centering
\includegraphics[width=0.5\textwidth]{fig_Oslofjorden_area}
\caption{The Oslo fjord. The position of the four tide gauges in Oslo, Oscarsborg, Viker, and Helgeroa are marked.}
\label{fig:area1}
\end{figure}

The Oslo fjord is located in the southeastern part of Norway with the main city of Norway in the innermost part of the fjord (Fig.~\ref{fig:area1}). The fjord lies in the most populated area of Norway and it is therefore important to gain knowledge on this fjord. The fjord has an interesting flow pattern due to several river outlets, thresholds, complex topography, storm situations in Skagerak, and atmospheric forcing. 
Even though the mean total tidal elevation is less than 20 cm in the Oslo fjord, the tidal currents are up to 1 m/s due to the narrow straits and thresholds.In connection with storm surge events, a maximum amplitude of 70 cm is observed. The Oslo fjord has an open, southern boundary towards Skagerak which lies in the eastern part of the North Sea. The circulation in the Skagerak is anticlockwise with brackish outflow from the Baltic Sea \cite[]{rodhe96,svendsen96}. 
The inner part of the fjord has two branches. The western branch is almost cut in two by a narrow, shallow, and long sill which is only 11 metes deep, 180 meters wide, and more than 1 km long. This sill causes a relatively strong tidal current, called Svelvikstraumen. The eastern branch also has a sill. Close to Oscarsborg there is a partly man made sill wich consist of an underwater barrier only 1-2 meters deep and extends halfway across the fjord from the western side. Towards the western side there is a natural sill of about 20 meters depth. North of the sill the maximum depth is more than 120 meters in both branches.  This makes the Oslo fjord peculiar among Norwegian fjords in that most of them have the sill at the entrance to the fjord.

\begin{figure}[!t]
\centering
\includegraphics[width=0.8\textwidth]{fig_Saltstraumen_area}
\caption{The Saltstraumen is a strong current in a narrow passage between the Salt fjord and the Skjerstrad fjord. The position of the tide gauge in Bod\o is marked.}
\label{fig:area2}
\end{figure}

The Saltfjord is located in the northern part of Norway (Fig.~\ref{fig:area2}). Saltstraumen is a small straight at the entrance of the Skjerstad fjord and is counted as the world's strongest tidal currents. The difference between the water level before the narrow straight and after the narrow straight can be up to one meter, causing water speeds up to 11 m/s \cite[]{eliassen01}.
The inner part of the fjord system, the Skjerstad fjord, is a deep glacially carved basin separated from the outer part, the Salt fjord, by a sill in the narrow current named Saltstraumen. This fjord system is generally deep, but the depth from the sill to the mouth of Saltfjorden is less than the depth of the inner part, Skjerstadfjorden. The sill in Saltstraumen is about 50 km from the head of the fjord system. The sill depth is 26 m and the depth in the basin inside the sill is more than 500 m: The inner basin and the sill area are connected by a channel which is much deeper and a little wider than Saltstraumen.

The Norwegian Mapping Authority, Hydrographic Service has 23 tide gauges along the Norwegian coast included one at Spitsbergen. Data is available from 1992 up to the current year. Four of the gauges are placed in the Oslo fjord  (Fig.~\ref{fig:area1}) and one close to the open boundary of the Salt fjord (Fig.~\ref{fig:area2}).

