\section{Introduction}

Tides are said to have the longest and most extensive history as a subject of scientific research among all altimetric corrections \cite[]{egbert94,cartwright77,hendershott81}. 
Already in the late 1940 \citep{unna47}, the tides was thought to be so well understood that two decades later \cite{munk66} apologised for attempting to improve the one geophysical prediction that worked tolerably well already. Since then numerous scientists have continued to improve the understanding and prediction of tides. 

The tides are especially challenging in coastal waters. Even though global ocean tide models have reached an impressive level of accuracy, there still remain outstanding challenges in particular in shallow water areas \citep{stammer14}. High-resolution global models cannot be expected to produce accurate tide predictions in coastal waters due to lack of sufficiently fine resolution of  bathymetric data and local tidal knowledge. Commonly coastal tides are one of the most dominant contributors to sea level variations in fjords \citep[e.g.][]{grabbe09}, and the dynamics of numerous fjords of different topography, climatology, are impacted by tides. Since tides affect the surface circulation in fjords, it is important to local communities and industries to enhance our knowledge of the tides. In addition, the tides are also driving forces for internal waves that influence the vertical mixing in sill fjords \citep{stigebrandt76}, and can generate zones of intensive mixing near shallow sills \citep{staal15}. As a consequence accurate knowledge of tides are important for understanding the hydrography and ecology of the marine coastal waters.

As in many regional models, the tides are often imposed only on the boundary of the fjord models \citep{gjevik89,carniello05,lynge13}. Close to the coastline, the tides vary over short distances, and shallow water constituents with quarter-diurnal periods are of importance \citep[e.g.][for the case of the Oslofjord]{trygg74}. 
Due to complex topography with narrow and shallow straits, the tides in the innermost parts of the Norwegian fjords are often shifted in phase and altered in amplitude compared to the tides in the open water outside the fjords. \cite{chen99} adjusted tidal forcing for the two-dimensional Princeton Ocean Model (POM) using a simple least square scheme with observations from tide gauges. A weighted variational formulation is proposed in order to modify the tidal forcing into hydrodynamic models of the open ocean using observations of water level along the coast \citep{bennett82}. \cite{zhang03} assimilated tide gauge data through gradients of the cost function in order to estimate tidal forcing for tidal simulation of the US East Coast using the POM model.

Here we present a method whereby the global tidal forcing may be adjusted to improve the forecast of local tides in three-dimensional, non-linear fjord models. The method is illustrated using the TPXO and Atlantic tidal atlases \citep{egbert94,egbert02} in the Regional Ocean Model System (ROMS) \citep{shchepetkin05,shchepetkin09,haidvogel08}. Two fjords in Norway with different topography and dynamics, namely the Oslofjord and the Saltfjord, are chosen when evaluating the method.
