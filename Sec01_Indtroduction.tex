\section{Introduction}

Tides are said to have the longest and most extensive history as a subject of scientific research among all altimetric corrections \cite[]{egbert94,cartwright77,hendershott81}. Already in 1947, the tides was thought to be so well understood that \cite{unna47} stated that the expressions must not be applied too literally, for that would imply that the tidal wave behave as it ought to do, which is not the case. Two decades later,  \cite{munk66} apologised for attempting to improve the one geophysical prediction that worked tolerably well already. Since then numerous scientists have continued to improve the understanding and prediction of tides. 

The tides are especially challenging in coastal waters. Even though global ocean tide models have reached an impressive level of accuracy, there still remain outstanding challenges especially in shallow-water areas \cite[]{stammer14}. 
High-resolution global models cannot be expected to produce accurate tide predictions in coastal waters due to lack of local bathymetric data and local tidal knowledge. 

Along the Norwegian coast tides are one of the dominant contributor to sea level variations \cite[]{grabbe09}. Numerous fjords with different topography, climatology, and dynamics are affected by tides. Tides affect the surface circulation in fjords, and for the local communities and industries it is important to improve knowledge of the tides. In addition, the tides are also the driving force for internal waves that affect the vertical mixing in sill fjords \cite[]{stigebrandt76}, and can generate zones of intensive mixing near shallow sills \cite[]{staal15}. As a consequence accurate knowledge of tides are important for understanding the hydrography and ecology of the marine coastal waters.

As in many regional models, the tides are often imposed only on the boundary of the fjord models \cite[]{gjevik89,carniello05,lynge13}. Close to the coastline, the tides vary over short distances, and shallow water constituents with quarter-diurnal periods are of importance, see \cite{trygg74} for the case of the Oslofjord. Due to complex topography with narrow and shallow straights, 
the tides in the innermost parts of the Norwegian fjords are often shifted in phase and altered in amplitude compared to the tides in the open water outside the fjords. 

\cite{chen99} adjusted tidal forcing for the two-dimensional Princeton Ocean Model (POM) using a simple least square scheme with observations from tide gauges. A weighted variational formulation is proposed in order to modify the tidal forcing into hydrodynamic models of the open ocean using observations of water level along the coast \cite[]{bennett82}. \cite[]{zhang03} assimilated tide gauge data through gradients of the cost function in order to estimate tidal forcing for tidal simulation of the US East Coast using the POM model. 

Here we propose a new and simple method on how to adjust the global tidal forcing to improve the forecast of tidal currents in three-dimensional non-linear fjord models. The method is illustrated using the TPXO and Atlantic tidal atlases \cite[]{egbert94,egbert02} in the Regional Ocean Model System (ROMS) \cite[]{shchepetkin05,shchepetkin09,haidvogel08}. Two fjords in Norway with different topography and dynamics are chosen as model areas. 
