\section{Introduction}

Tides are said to have the longest and most extensive history as a subject of scientific research among all altimetry corrections \cite[]{egbert94,cartwright77,hendershott81}. 
Already in the late 1940 \citep{unna47}, tides was thought to be so well understood that two decades later \cite{munk66} apologised for attempting to improve the one geophysical prediction that worked tolerably well already. Since then numerous scientists have continued to improve the understanding and prediction of tides. 

Even though global ocean tidal models have reached an impressive level of accuracy, there still remain outstanding challenges in shallow coastal water areas \citep{stammer14}. High-resolution global models cannot be expected to produce accurate tide predictions in coastal waters due to lack of sufficiently fine resolution of bathymetric data and local tidal knowledge. Commonly coastal tides are one of the most dominant contributors to sea level variations in fjords \citep[e.g.][]{grabbe09}, and the dynamics of numerous fjords of different topography and coastline geometry, are impacted differently by tides.

Since tides affect the surface circulation in fjords, it is important to local communities and industries to enhance our knowledge of the tides. In addition, the tides are also driving forces for internal waves that influence the vertical mixing in sill fjords \citep{stigebrandt76,staal16}, and can generate zones of intensive mixing near shallow sills \citep{staal15}. As a consequence, accurate knowledge of tides is important for understanding the hydrography and ecology of the marine coastal waters.

Commonly, tides are often imposed on the boundary of the fjord models only \citep{gjevik89,carniello05,lynge13}. In the shallower waters close to the coastline the tides vary over short distances, and the shallow water constituents with quarter-diurnal periods are of importance \citep[e.g.]{trygg74}. 
Due to complex topography with narrow and shallow straits, the tides in the innermost parts of the fjords are often shifted in phase and altered in amplitude compared to the tides outside of the fjords. 

Previous studies by for instance \cite{chen99} adjusted tidal forcing for the two-dimensional Princeton Ocean Model (POM) using a simple least square scheme with observations from tide gauges. Also, a weighted variational formulation is proposed to modify the tidal forcing into hydrodynamic models of the open ocean using observations of water level along the coast \citep{bennett82}. Furthermore, \cite{zhang03} assimilated tide gauge data through gradients of the cost function in order to estimate tidal forcing for tidal simulation of the US East Coast using the POM model.

We present a method whereby the tidal forcing from a global or regional tidal model may be adjusted to improve the forecast of local tides in three-dimensional, non-linear fjord models. 
The method is applied to a model of the Oslofjord in which a curvilinear version of the Regional Ocean Model System (ROMS) \citep{shchepetkin05,shchepetkin09,haidvogel08} adapted to the Oslofjord is used as the tidal model. The method global tidal forcing is illustrated using the TPXO Atlantic tidal atlas \citep{egbert94,egbert02}.
We first run the Oslofjord model using the raw global tidal forcing on its open boundary. Comparing modelled and observed time series of water level at a tidal gauge station close to the open boundary of the Oslofjord model, a factor for the amplitude and a shift in phase is computed. The amplitude factor and the phase shift are then applied to produce adjusted tidal forcing at the open boundary. Next we rerun the Oslofjord model using the adjusted tidal forcing. The results from the two runs are then compared to independent observations inside the fjord in terms of amplitude and phases of the various tidal components, the total water level and the currents.

An improvement in the tidal forcing at the open boundary is expected to have a positive effect throughout the model domain. The quality of the tidal representation in the inner domain, however, depends also on the representation of the tidal propagation. In turn the representation of the tidal propagation is highly influenced by the topography and the bottom friction in shallow waters. In order to improve the shallow water constituents in the innermost parts of the domain, a better representation of the tidal propagation might be needed. This is outside the scope of this study. 
