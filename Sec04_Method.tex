\section{Method}

Commonly open ocean tidal information from a coarser model is simply interpolated onto the finer model grid of the regional model. This approach might not be sufficient for a fjord model because the many small and large islands inside the fjord and the fjord's irregular coastline are not properly resolved in for example the Atlantic TPXO tidal solutions. In order to avoid this problem it has previously been common to apply tidal amplitude and phase from observed water levels close to the boundary \cite[i.e.]{foreman90,svendsen96,lynge13}. 

Here we take a somewhat different approach in that we propose a method on how to adjust the global tidal forcing in terms of both elevation and depth integrated currents. The method is straightforward. In summary, the fjord model is first run using raw tidal forcing on the open boundary of the fjord. Comparing modelled and observed time series of tidal water level at a tidal gauge station close to the open boundary of the model, a factor for the amplitude and a shift in phase is computed. The amplitude factor and the phase shift are then applied to produce adjusted tidal forcing at the open boundary. Next the model is rerun using the adjusted tidal forcing. The results from the two runs are then compared to independent observations inside the fjord in terms of amplitudes and phases of the various tidal components, the total tidal water level, and the depth integrated tidal currents. To evaluate the method we present results for the Oslofjord, Norway using the Oslofjord model as described in Section \ref{sec:model_setup}. 

One of the requirements for the proposed method to work properly is that the phase shift along the boundary is negligible. This is achieved if the open boundary is as parallel as possible to the face of the tidal wave, or if the width of the open boundary is sufficiently narrow. Another requirement is that there must be at least one water level observations available at, or close to, the open boundary.
The open boundary for the Oslofjord is less than 50 km wide, which is relatively narrow. There are two tidal gauge stations close to the open boundary, Helgeroa and Viker (Fig. \ref{fig:area1}). However, tidal analysis reveals that the harmonic components from the Helgeroa stations differ with less than 0.5 cm for the amplitudes and 1 degree for the phases \cite[]{tide16} from the Viker station. Therefore, it is deemed sufficient in this study to include one station only. We emphasise though that for fjords exhibiting wider openings it might be necessary to apply more than one station. 

The application to the Oslofjord consists of two steps. In the first step we first extract the relevant tidal forcing from the Atlantic TPXO global model which has a resolution of 1/30 degree. It is then imposed at the open boundaries of the Oslofjord model, and a 150 day simulation starting 1 April 2014 is performed (henceforth Run 1). Next, time series of observed and modelled water level from the Viker tidal gauge station are extracted and analysed using the T\_Tide package described by \cite{pawlowicz02}. The standard set of 69 components are applied in the analyses, but only the components included in or derived from the global tidal forcing are considered.
The observations are compared to the results from Run 1 at the Viker station in terms of their tidal amplitudes and corresponding phases. An amplitude factor, $c^{(n)}$, and a phase shift, $\triangle \phi^{(n)}$, for each tidal component $n$ is then computed according to
\begin{eqnarray}
c^{(n)} &=& \frac{a^{(n)}_{obs}}{a^{(n)}_{sim}}, \\
\triangle \phi^{(n)} &=& \phi^{(n)}_{obs} - \phi^{(n)}_{sim},
\end{eqnarray}
where $a^{(n)}_{obs}$ and $a^{(n)}_{sim}$ are the observed and modelled amplitudes and $\phi^{(n)}_{obs}$ and $\phi^{(n)}_{sim}$ are the observed and modelled phases, respectively. 

In the second step we adjust the amplitudes and phases at the boundary using the above computed factors and phase shifts. The adjusted amplitudes and phases are simply given by
\begin{eqnarray}
a^{(n)}_{i2} &=& a^{(n)}_{i1} c^{(n)}, \\
\phi^{(n)}_{i2} &=& \phi^{(n)}_{i1} + \triangle \phi^{(n)},
\end{eqnarray}
where $a^{(n)}_{i1}$ and $\phi^{(n)}_{i1}$ are the amplitude and phase originally imposed on grid cell $i$ along the boundary in Run 1. Modified major and minor amplitude and phase for the tidal current are adjusted in the same way using the amplitude factor and phase shift calculated from the water level. In this we assume a linear relation between water level and currents. The inclination angles are not adjusted.

Next the model is rerun using the adjusted tidal forcing (henceforth Run 2) for the same period as Run 1 and the results are analysed the same way as for Run 1. Finally, the results from both runs are compared with the observed tidal water level and depth integrated currents (Section \ref{sec:discuss}). 

